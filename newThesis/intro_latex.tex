\chapter{Introduction to \LaTeX}
\LaTeX{} is a text formatting system that dates back to the 1980s. With 
\LaTeX{} you create your document by editing a text input file using a
program such as EMACS, \verb+vi+, or another editor. You then give
this input file to \LaTeX{} (or, more accurately, to a program called
\verb+pdflatex+). This program then literally \emph{compiles} your
input file into a PDF file.

There are many reasons that you may wish to use \LaTeX{} for preparing
a technical report, journal article, or master's thesis:

\begin{itemize}
\item \LaTeX{} allows you to directly include other files at the time
  that the PDF is created. This makes it easy to automatically
  incorporate the results of experiments or graphing programs directly
  into your final document, without having to retype anything.
\item Because the \LaTeX{} input file is plain ASCII, you can store
  your document using a source code control system such as
  Subversion\cite{subversion}. This allows multiple people to work on
  the same document at the same time; Subversion automatically merges
  the changes together.
\item The Bibtex bibliography system automatically maintains your
  citations and bibliography. The citation format is maintained
  separately from the citation contents, allowing you to easily change
  citation styles when submitting to different conferences or journals.
\item Unlike Word, \LaTeX{} gives you precise control over the
  placement of the text on the page. You can easily make global changes
  to your document and have them reflected everywhere. 
\item \LaTeX{} is free software and runs on PCs, Macs, and Unix
  systems. This means that you can produce your documents on
  practically every computer you have, without having to purchase
  anything else.
\end{itemize}

There are many good reference books and online tutorials for \LaTeX
out there. The purpose of this document is not to duplicate that work,
but to provide you with the minimum amount of information that you
require to use \LaTeX{} to produce a master's thesis or technical
report at NPS.


\section{Installation}
Before you can use \LaTeX, you will need to install two critical
pieces of software:
\begin{enumerate}
\item The \LaTeX environment itself.
\item A program for editing the |.tex| input files.
\end{enumerate}

Here once again there are many options that you have. For both \LaTeX
and text editors there are both Free Open Source and commercial
distributions. This document makes specific recommendations that were
known to work as of the document's date of publication. You are free
to explore on your own as well.


\subsection{Installation on MacOS 10.5 and above}

There are many ways to get \LaTeX running on the Mac. The most common
are:

\begin{enumerate}
\item Download a working installer for the most recent distribution
  from \url{http://www.tug.org/mactex}.
\item Install it from sources using \url{http://www.macports.org}.
\item Install it using the \textit{i-Installer}.
\end{enumerate}

\subsubsection{Installing from TuG}

TUG's MacTeX distribution will install \LaTeX in the
|/usr/texbin/pdflatex| directory and will update your startup files to
include this directory in your path. If you chose this strategy, be
sure to click ``Customize'' in the installer and select that all of
the optional packages be installed:

\subsubsection{Installing with MacPorts}

MacPorts will download and compile the \LaTeX distribution from
sources and then install it in the |/usr/local/bin| directory. This
approach involes more work on your part to get going, but less work in
the long run, as you can use the MacPorts system to automatically
install many other open source software that might be of interest, and you
can easily upgrade the installation at a later point in time.

MacPorts also requires that the Apple XCode system be installed. You should
really have this installed anyway, since you need XCode to compile
programs. 

\begin{enumerate}
\item Verify that XCode is installed by opening a Terminal and typing
  |gcc|. 

If you see something that looks like this, you are good to go:
\begin{Verbatim}
$ gcc
i686-apple-darwin10-gcc-4.2.1: no input files
$ 
\end{Verbatim}

If you see a ``command not found'' error, you need to install XCode:
\begin{Verbatim}
$ gcc
-bash: gcc: command not found
$ 
\end{Verbatim}

You can install from your installation CD or download it
from \url{http://developer.apple.com} (you will need to create an
Apple Developer account to perform the download.)

\item Visit \url{http://www.macports.org} and click the Download
  button, which you will find in the upper-right hand corner of the
  website.
\item Select to download the |dmg| file.
\item Open the |dmg| file. Run and install the MacPorts package. You
  will need to type your password.
\item Now you need to ``update'' the MacPorts library, and then
  install the |teTeX| distribution (which includes \LaTeX and several
  other tools). Open a Terminal window and type this command:
\begin{Verbatim}
# sudo port selfupdate
# sudo port install teTeX
\end{Verbatim}

You are free to download and install a tool for editing the |.tex| and
|.bib| files. However, you can also edit these files using EMACS, an
editor that is built-in to MacOS. 

\subsection{Installing on Ubuntu Linux or Debian Linux}
Use this command:

\begin{Verbatim}
% sudo apt-get install texlive-latex3
\end{Verbatim}

We have noticed that the command occasionally fails. If it does, try
it again. If that still doesn't work

\subsection{Installation on Windows}

\section{Running \LaTeX}
\LaTeX{} is actually a set of programs. For creating a thesis at NPS
you will use four programs:
\begin{description}
\item{\texttt{pdflatex}} This program read the input file (\eg
  \verb+thesis.tex+) and produces a PDF file (\eg \verb+thesis.pdf+)
  as an output. This program also produces a number of intermediate
  file (\verb+thesis.aux+, \verb+thesis.bbl+, \verb+thesis.toc+,
  \emph{et. cetera}.)
\item{\texttt{bibtex}} This program reads the \verb+thesis.bbl+ file and
  produces a bibliography in a file called \verb+thesis.bst+ which
  includes the bibliography. The \verb+thesis.bst+ then gets read the
  next time \verb+pdflatex+ is run.
\item{\texttt{authorindex.pl}} a program in perl that produces the
  author index from the \verb+thesis.bbl+ file. The authorindex is
  saved in the file \verb+thesis.ain+.
\item{\texttt{fixbbl.py}} It turns out that there is a bug in BibTeX
  which causes URLs longer than 53 characters to be improperly
  split. This program unsplits them.
\end{description}

Normally you would run \LaTeX{} with a three step process:
\begin{itemize}
\item \verb+pdflatex thesis+
\item \verb+bibtex thesis+
\item python fixbbl.py thesis
\item 
\item \verb+pdflatex thesis+
\item \verb+pdflatex thesis+
\end{itemize}

\section{Basic \LaTeX}
Here is a simple document:
\begin{Verbatim}
\documentclass{article}
\begin{document}
Hello World!
\end{document}
\end{Verbatim}

Normally with \LaTeX you just type text. Leave a blank line between
each paragraph. \LaTeX then formats it into beautiful
paragraphs. \LaTeX will ignore the space at the beginning of each line.

So if you type this:

\begin{Verbatim}
In December 1951, in a move virtually unparalleled in the history of
academe, the Postgraduate School moved lock, stock and wind tunnel
across the nation, establishing its current campus in Monterey,
Calif. The coast-to-coast move involved 500 students, about 100
faculty and staff and thousands of pounds of books and research
equipment. Rear Adm. Ernest Edward Herrmann supervised the move that
pumped new vitality into the Navy's efforts to advance naval science
and technology.
\end{Verbatim}

\LaTeX will format it to look like this:

\begin{quotation}
In December 1951, in a move virtually unparalleled in the history of
academe, the Postgraduate School moved lock, stock and wind tunnel
across the nation, establishing its current campus in Monterey,
Calif. The coast-to-coast move involved 500 students, about 100
faculty and staff and thousands of pounds of books and research
equipment. Rear Adm. Ernest Edward Herrmann supervised the move that
pumped new vitality into the Navy's efforts to advance naval science
and technology.
\end{quotation}

\subsection{Typing Quotes}

To type quotes, you should not use the double-quote character. Instad,
\LaTeX uses the back quote (\verb+`+) and the forward quote (\verb+'+)
to type quotes.

\begin{tabular}{ll}
To get this & type this \\
don't & \verb+don't+ \\
 3'2''  & \verb+3'2'+ '\\
``this'' & \verb+``this''+ \\
`is' & \verb+`is'+ \\
``\,`special'\,'' & \verb+``\,`special'\,''+ \\
\end{tabular}

The last one is a bit confusing, but don't worry, everything will make
sense in a bit.

\subsection{Special Characters}

Unlike Microsoft Word and other programs, \LaTeX uses special
characters embedded in your text to control formatting. The most
common of these characters is the backslash ($\backslash$). To produce
each of these characters you must use a special sequence that begins,
strangely enough, with a backslash.

The table below shows the ten special characters.

\begin{tabular}{ll|ll}
To get this & type this & To get this & type this \\
\$ & \verb+\$+ & \& & \verb+\&+ \\
\{ & \verb+\{+ & \} & \verb+\}+ \\
\% & \verb+\%+ & \_ & \verb+\_+ \\
\# & \verb+\#+ & \^{} & \verb+\^{}+ \\
\~{} & \verb+\~{}+ & $\backslash$ & \verb+$\backslash$+ \\
\end{tabular}

The good news is that you typically don't need these characters very
much when preparing technical documents.

\subsection{Accented Vowels}
For the most part \LaTeX builds accented vowls by combining an accent character with a
vowel:

\begin{tabular}{cl|cl}
To get this & Type this & To get this & type this\\ 
\'{o} & \verb+\'{o}+ & \~{o} & \verb+\~{o}+ \\
\'{o} & \verb+\'{o}+ & \={o} & \verb+\={o}+ \\
\^{o} & \verb+\^{o}+ & \.{o} & \verb+\.{o}+ \\
\"{o} & \verb+\"{o}+ & \d{o} & \verb+\d{o}+ \\
\c{o} & \verb+\c{o}+ & \u{o} & \verb+\u{o}+ \\
\b{o} & \verb+\b{o}+ \\
\end{tabular}


However, there are some special accented characters that are just specials:

\begin{tabular}{ll|ll}
To get this & Type this\\
\aa & \verb+\aa+ & \AA & \verb+\AA+ \\
\end{tabular}

Sometimes you will need to use dotless characters. You can get them
with these sequences:

\begin{tabular}{ll|ll}
\i & \verb+\i+ & \j & \verb+\j+ \\
\o & \verb+\o+ & \O & \verb+\O+ \\
\end{tabular}

These sequences will give you slashes and daggers:
\begin{tabular}{ll|ll}
\l & \verb+\l+ & \L & \verb+\L+ \\
\dag & \verb+\dag+ & \ss & \verb+\ss+ \\
\P & \verb+\P+ & \pounds & \verb+\pounds+ \\
\end{tabular}


\subsection{Special Characters}


\begin{tabular}{ll|ll}
type this & To get this &type this & To get this\\
?` & \verb+?`+ & \copyright & \verb+\copyright+ \\
\oe & \verb+\oe+ & \OE & \verb+\OE+ \\
!` & \verb+!`+ & \S & \verb+\S+ \\
\v{o} & \verb+\v{o}+ & \H{o} & \verb+\H{o}+ \\
\t{oo} & \verb+\t{oo}+ & \ae & \verb+\ae+ \\
\AE & \verb+\AE+ & \l & \verb+\l+ \\
\dag & \verb+\dag+ & \P & \verb+\P+ \\
\ss & \verb+\ss+ & \L & \verb+\L+ \\
\ddag & \verb+\ddag+ & \pounds & \verb+\pounds+ \\
\end{tabular}

\subsection{Changing font size}
\begin{tabular}{ll}
Command & Example\\
\verb+\tiny+ & {\tiny this is 6 point font} \\
\verb+\scriptsize+ & 8\\
\footnotesize & 10\\
\small & 11\\
\normalsize & 12\\
\large & 14\\
\Large & 17\\
\LARGE & 20\\
\huge & 25\\
\Huge & 25\\
\end{tabular}

\subsection{Changing font style}
\begin{tabular}{ll}
Command & Example\\
{\rm Roman} \\
\textbf{This is bold}\\
\texttt{This is typewriter}\\
\textsc{This is small capitals}\\
\textsl{This is slanted}\\
\textit{Itallics}\\
\emph{This is emphasized}\\
$\cal CALLIGRAPHIC$\\
{\boldmath $\cal BOLD CALIGRAPHICS$}\\
\end{tabular}

\subsection{choosing an arbitrary font}

\subsection{Controlling Spaces}

Use:

 \verb+\,+ (a slash followed by a comma) to produce a small space.
 \verb+\ + (a slash followed by a space) to produce a standard word  space
 \verb+\@+ (a slash followed by an at sign) to produce a standard
 intersentence space.




\begin{tabular}{ll}
Type this & To get this \\
\verb+``One Way''+ & ``One Way'' \\
\end{tabular}

\subsection{Controlling Line Breaks}
You can force a blank line with: \verb+\\+

Like this:

This is\\
a test

You can insert an arbitrary amount of space with the optional
argument:

This is\\[3pt]
a test

You can add or remove space on a page with
\verb+\enlargethispage+. For example, to squeeze a second line, try
this:

\begin{verbatim}
    \enlargethispage{1pc}
\end{verbatim}

You can insert verticale or horizontal space (or take it away) with

\verb+\vspace{1in}+
\verb+\hspace{1in}+

You can draw a box around text with:

\verb+\framebox[width]{textstring}+

The width is optional.

Centering:

\begin{center}
This is centered.
\end{center}


\subsection{Lists}

There are three kinds of lists that you may wish to make:

\begin{description}
\item[description] lists are used for definitions (like this).
 \item[enumerated] lists are lists where each item is numbered and the
  ordering is relevant, like the steps of a recipe. 
\item[itemized] lists are lists where each item is of equal
  importance.
\end{description}

Lists are implemented as \LaTeX environments, which means that they
begin with a \verb+\begin{+\emph{listname}\verb+}+ and end with an
\verb+\end{+\emph{listname}\verb+}+.

\begin{minipage}{.45\textwidth}
\begin{Verbatim}
\begin{enumerate}
\item Wake up.
\item Go to work.
\item Go home.
\item Go to sleep.
\item Repeat.
\end{enumerate}
\end{Verbatim}
\end{minipage}
\hfill
\begin{minipage}{.45\textwidth}
\begin{enumerate}
\item Wake up.
\item Go to work.
\item Go home.
\item Go to sleep.
\item Repeat.
\end{enumerate}
\end{minipage}


\subsection{Tables}
\subsection{Math}
\subsection{Graphics}
This section briefly describes graphics and \LaTeX. For alternative
treatments we recommend H\"oppner's  ``Strategies for including graphics in \LaTeX
documents''\cite{strategies}, and the book TK.

There are many graphic file formats and practically all of them can be
embedded within a \LaTeX document. Unfortunately, including graphics
is complicated by the fact that there are two flavors of \LaTeX and
they both require different graphic file inputs.



There are two kinds graphics that you can include in a \LaTeX
document: vector graphics and bitmaps:
\begin{description}
\item[Vector graphics] are graphical objects that contain commands for
  rendering an illustration---the graphic describes \emph{how}
  something is to be drawn. Typical examples of vector graphics are
  text and 
\end{description}

\section{BiBTeX}
% http://stefaanlippens.net/bibentry
