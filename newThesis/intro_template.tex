\section{Getting \LaTeX}
Get a copy of the \texttt{npsthesis.tar.gz} distribution from
\url{http://faculty.nps.edu/slgarfin/npsthesis.tar.gz}. Unpack this
into a directory on your computer; \tabref{files} shows the important
files that you will fine.

\newcommand{\fileinfo}[2]{\texttt{#1} & #2\\}
\begin{table}[h]
\frame{
\begin{tabularx}{\textwidth}{lX}
\fileinfo{Makefile}{The Makefile to make the thesis}
\fileinfo{authorindex.*}{The \LaTeX authorindex package, for making
the Referenced Authors page.}
\fileinfo{chngcntr.sty}{The \texttt{chngcntr} package, for changing
the way that \LaTeX displays its counters.}
\fileinfo{fixbbl.py}{A python program that removes the breaks in the
\texttt{.bbl} file inserted by BibTex.}
\fileinfo{nps-plain.bst}{A BibTex style file that makes references in a style that is acceptable to NPS.}
\fileinfo{nps-plain-unsorted.bst}{A BibTex style file that makes references in a style that is acceptable to NPS.}
\fileinfo{nps-plain-classified.bst}{A BibTex style file that makes references in a style that is acceptable to NPS.}
\fileinfo{nps-plain-classified-unsorted.bst}{A BibTex style file that makes references in a style that is acceptable to NPS.}
\fileinfo{nps\_logo\_3clr\_cymk.pdf}{NPS Logo, 3 color}
\fileinfo{nps\_logo\_black\_cymk.pdf}{NPS Logo, black and white}
\fileinfo{thesis.bib}{Your thesis bibliography file}
\fileinfo{thesis.tex}{Your thesis \LaTeX source}
\end{tabularx}}
\caption{Files included with your distribution.}\label{files}
\end{table}

\begin{table}
\frame{\begin{tabularx}{\textwidth}{lX}
\fileinfo{\verb+\chapref+}{Chapter reference that formats as ``Chapter 3''}
\fileinfo{\verb+\chapvref+}{Chapter reference that formats as ``Chapter 3 on page 4''}
\fileinfo{\verb+\secref+}{Section reference that formats as ``Section
  3.'' You can use this for sections, subsections, and so on.}
\fileinfo{\verb+\secvref+}{Section reference that formats as ``Section 3 on page 4''}
\fileinfo{\verb+\figref+}{Figure reference that formats as ``Figure 3''}
\fileinfo{\verb+\figvref+}{Figure reference that formats as ``Figure 3 on page 4''}
\fileinfo{\verb+\tabref+}{Table reference that formats as ``Table 3''}
\fileinfo{\verb+\tabvref+}{Table reference that formats as ``Table 3 on page 4''}
\fileinfo{\verb+\appref+}{Appendix reference that formats as ``Appendix 3''}
\fileinfo{\verb+\appvref+}{Appendix reference that formats as ``Appendix 3 on page 4''}
\end{tabularx}}
\caption{Reference commands included in the \texttt{thesis.tex} demo file.}\label{refcommands}
\end{table}

To create your thesis, start with the file \texttt{thesis.tex}. 
\texttt{thesis.bib} 

\section{Tables and Figures}
Tables and figures are floating objects that \LaTeX moves around as
necessary to make your thesis look better. Tables are inserted with
the \verb+\begin{table}+ command while figures are inserted with
\verb+\begin{figure}+. Here are some rules to consider:

\begin{itemize}
\item Every table and figure should have a caption, created with the
  \verb+\caption{text}+ command.
\item Every table and figure should have a unique label, created with
  the \verb+\label{marker}+ command.
\item Every table and figure should be referred to in the main body of
  your text. \LaTeX provides a command called \verb+\ref{marker}+;
  this template provides additional commands \verb+\tabref{marker}+
  and \verb+\figref{marker}+. All of the reference commands are shown
  in \tabvref{refcommands}.
\item Do not assume that figures will be on the same text as your
  page. Always refer to the 
\end{itemize}

\section{Including Photos and Figures}
This section shows how you can easily include photos. 

Using the \verb+\sgraphic{filename}{caption}+ command you bring in a
photo from a given filename and give it a caption. The filename is
then automatically set up as \LaTeX cross-reference. Use the
\verb+\figref{tag}+ command to get an in-paragraph
reference. \figref{demos/demo_art/home_topimg} shows an example of this.

\sgraphic{demos/demo_art/home_topimg}{Banner from the top of the NPS web site}

The \verb+\twofigures{width1}{image1}{caption1}{width2}{image2}{caption2}+
macro allows you to have two figures side-by-side, as shown in
\figref{demos/demo_art/photo1} and \figref{art/photo2}.

There are a large number of these layout macros at the end
of\verb+npsthesis.cls+ --- give them a look!

\twofigures{2.5in}{demos/demo_art/photo1}{A photo from the NPS web site}
           {2.5in}{demos/demo_art/photo2}{A second photo from the NPS
             web site.}

\section{Going Further}
If you are interested, feel free to review the file
\verb+npsthesis.cls+. A great deal of effort has gone into making this
file both readable and understandable. You will find additional
commands in this file and you may even have thoughts on changes to
make. Please let us know what you come up with!

You may find the following packages useful:

\verb+multirow+ -- Allows a single table cell to extend to multiple
rows.
\verb+ifthen+ --- allows you to put conditions in your thesis. It's a
bit easier than using the \verb+if+ that's built in to \TeX.



