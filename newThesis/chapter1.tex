\chapter{Introduction}
\paragraph*{} Mobile devices have become ubiquitous throughout the world.  Mobile phones, in particular, have evolved from large clumsy devices, available to only a select few, to miniature computers in the hands of millions of people.  Communications on mobile devices encompasses more than just phone calls.  Short messages using SMS and Twitter services are used in increasing numbers everyday.  SMS usage has grown from 81 billion messages in 2005 to 2.1 trillion messages in 2010. \cite{_u.s._????} Twitter posts have grown from 5,000 tweets per day in 2007 to 35 million tweets per day in 2010.\cite{_twitter_????} Email, once solely reserved for personal computers and workstations, is now widely used from mobile phones and tablets.  34 percent of mobile phone users sent emails from their phones in 2010, up from 25 percent in 2009. \cite{smith_mobile_2010}
\paragraph*{} While the versatile and always-on communications provided by mobile devices has been a boon to society, it has also been a powerful tool for terrorists, child predators, and other criminals.  Disposable phones make nefarious communications anonymous and bad people more difficult to find.  To combat anonymity, author detection tools capable of analyzing text communications, short messages and e-mail, of mobile devices is needed.  Reliable and near-real-time author detection could provide a continual means of tracking a person of interest and their mobile device.

\section{Using Mobile Devices to Locate Persons of Interest}
\paragraph*{} Simply identifying a mobile device does not identify an author.  Phones and tablets can be stolen, swapped, or thrown away in the case of disposable phones. Effectively using mobile communications to detect authors is a multi-step process.  First, the communications must be gathered through some process of eavesdropping. Second, the gathered communications must be processed to detect authors.  Lastly, some form of notification must be sent to the interested parties when an author is detected.
\paragraph*{} Eavesdropping on text communications across billions of cell phones is difficult at best.  With an estimated 2.1 trillion SMS text messages sent in 2010, processing the massive amount of data created by such an eavesdropping capability is overwhelming for a central processing facility.  With the growth of other short message services like Twitter, text messaging on mobile devices is only growing more prevalent. Central processing of data this massive can create a severe bottleneck.
\paragraph*{} Assuming that a covert method of delivering software to a mobile device is available, we investigate a different approach: to decentralize author detection for mobile text communications. In short, empower mobile phones to process text data for persons of interest on the mobile phone itself, not at a central processing facility.  Whether the intent is to screen the text messages of a teen for known child predators or to locate terrorists in a combat environment using cell phones to coordinate attacks, the computing ability of millions of cell phone processors is a powerful resource to tap.
\paragraph*{} The challenge with author detection on a mobile device is coping with very limited resources.  Even though 2011's smart phones are much more powerful than their predecessors, CPU speed and quantity are not on par with a high performance computing facility, the current domain of author attribution methods.  Volatile memory on a smart phone has grown as well, but many mobile operating systems such as Android impose severe limits on the allowed heap space.  To detect authors on a mobile device, the combination of classification methods, feature types (e.g. 1-grams, 2-grams, gappy bigrams, character n-grams), and vocabularies must be selected to optimize accuracy within the particular resource constraints of these devices. Storage, processing, and even power requirements must be taken into account. To describe the size impact of models, vocabularies, smoothing files etc. on a mobile device, the term storage requirement is used to collect the sizes of all author detection tools which need to be installed on the mobile device.  While the amount of volatile memory required for a particular storage requirement is not equal, the storage requirement is a relative indication of the volatile memory required.  A large storage requirement will create a larger volatile memory requirement.  Likewise a small storage requirement will create a smaller volatile memory requirement.

\section{Research Questions}
\paragraph*{} This thesis asks one basic question: can author detection be accomplished on a mobile device?  To answer that question, several supporting questions must be answered first: 
\begin{itemize}
	\item For the two dominant mobile phone text communication mediums, short message and e-mail, what combination of classification method and feature type provides the best accuracy?
	\item What is the storage requirement for each combination of method and feature type, and, hence, the best combination given limited operating resources?
	\item What is the classification accuracy versus storage requirement for each classification method and feature type?
	\item Does the relative prolificity of each author in a detection group significantly affect accuracy?
	\item Does a highly effective method-feature type combination exist with a small enough storage requirement to practically execute on a mobile device?
\end{itemize}

\paragraph*{} To answer these research questions, two corpora will be used as test data: the Enron E-mail Corpus \cite{brants_web_2006} and the NPS Twitter Short Message Corpus \cite{boutwell_sarah_author_2011}.  The Enron E-mail Corpus will be treated as a representative sample of e-mail communications.  The NPS Twitter Short Message Corpus will be used as a representative sample of short messages.  Since Twitter has identical character limits (140 characters, very short) to SMS messages, the NPS Twitter Short Message Corpus will be considered representative of both SMS and Twitter communications, although this thesis does not verify the veracity of this assumption.

\paragraph*{} To account for the widely varied nature of English language use in e-mail and short messages, the Google Web1T corpus \cite{brants_web_2006} will be used as a vocabulary to build models for this thesis.  This will provide a language reference populated with standard English as well as the evolving language habits on Internet and mobile device users.


\section{Significant Findings}
\paragraph*{} The testing of all classification method, feature type, and vocabulary combinations resulted in 19,782 tests producing 286,050 measurements and 19,782 measurements for average accuracy.  Analysis of these results finds:
\begin{itemize} 
	\item The method-feature type combination that suited mobile devices best for the Enron E-mail Corpus was Support Vector Machine classification using 1-grams as a feature type and no reference to the Google Web1T Corpus for vocabulary.  This combination produced an average accuracy of 77.4\% and average f-score of .6257 requiring 4.83MB (i.e. a feasible amount) of storage on the device.
	\item The method-feature type combination that suited mobile devices best for the Twitter Short Message Corpus was Support Vector Method using Gappy Bigrams with a word distance of 3 and no reference to the Google Web1T Corpus for vocabulary. This combination produce an average accuracy of 52.0\% and average f-score of 0.4820 requiring 3.59MB of storage on the device.
	\item Very prolific authors were detected with greater accuracy and f-score than less prolific authors, even when a prolific author was grouped with other prolific authors.
	\item Author detection accuracy and f-score, in the Enron E-mail Corpus was significantly higher than in the Twitter Short Message Corpus. However, it was not clear from the results if this disparity in accuracy is due to language differences between e-mail and short message or due to having a large amount of e-mail text compared to the amount of short message text.
	\item Similarly prolific authors had lower accuracies, but higher f-scores than dissimilarly prolific authors.    
	\item Storage requirements for many of the model-feature combinations were too large for use on a mobile device.  The most powerful method-feature combinations often had storage requirements above 2GB.
	\item There is a small number of method-feature combinations that can meet the storage limitations of a mobile device and still produce accuracies higher than MLE for author detection.  Whether these accuracies are sufficiently high for practical application is left for future work.
\end{itemize}
	

\section{Thesis Structure}
\paragraph*{} This thesis is organized as follows:
\begin{itemize}
\item Chapter I covers the motivation for this research, the research questions being answered in this thesis, and key findings of the research conducted.
\item Chapter II discusses prior work in authorship detection, machine learning, the corpora used, details of the Google Web1T corpus, and hashing strategies for managing the Google Web1T corpus.
\item Chapter III describes the combinations of classification methods, feature types, and vocabularies used during experimentation.  The limitations of the experimentation approach are discussed along with the metrics to be used to measure author detection performance.
\item Chapter V contains conclusions drawn from the results and recommended future work.
\end{itemize}

